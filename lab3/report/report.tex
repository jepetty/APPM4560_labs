\documentclass[11pt, oneside]{article}
\usepackage[margin=1in]{geometry}                % See geometry.pdf to learn the layout options. There are lots.
\geometry{letterpaper}                   % ... or a4paper or a5paper or ... 
%\geometry{landscape}                % Activate for for rotated page geometry
%\usepackage[parfill]{parskip}    % Activate to begin paragraphs with an empty line rather than an indent
\usepackage{graphicx}
\usepackage{amssymb}

\setcounter{secnumdepth}{3}

\title{APPM 4560 Laboratory 3 Report}
\author{Rhys Olsen\\
\texttt{rhys.olsen@colorado.edu}
 \and Jessica Petty\\
 \texttt{jessica.petty@colorado.edu}
 }
\date{November 16, 2016}

\begin{document}
\maketitle
\section{Simulating a Single Server Queue}
\subsection{Stationary Distribution of M/M/1-queue}
For an M/M/1 queue to have a stationary distribution, it is a necessary and sufficient condition that its arrival rate be less than its service. In the case of the queue provided, this means that $\lambda < \mu$.

\subsection{Distribution of a Stationary Queue}
When $X$ is stationary, this is equivalent to saying that it is in equilibrium. We know that an M/M/1-queue in equilibrium will have a geometric distribution of items inside. Now, given that $X_T$ is the number of items in the queue at time $T$, we can then say that the distribution of $X_T$ is geometric with success parameter $\frac{\lambda}{\mu}$.

\subsection{Busy Server}
When $X$ is stationary, the fraction of time that the server is \textit{not} busy is equivalent to the value $\pi(0)$. Conceptually, this is because the stationary distribution $\pi(x)$ represents the fraction of time that the queue has $x$ items in it. Clearly, if $x > 0$, this means the server is busy, as there are items in the queue, and when $x=0$ the server cannot be busy because there are no items in the queue, either waiting or being served.

For an M/M/1-queue, $\pi(0)=1-\frac{\lambda}{\mu}$. (PROOF can be found in notes if we decide it is necessary). Therefore, the fraction of time that the server \textit{is} busy is $1-\pi(0)=1-(1-\frac{\lambda}{\mu})=\frac{\lambda}{\mu}$

\subsection{}
\subsection{}
\subsection{}
\subsection{}
\subsection{}
\subsection{}
\subsection{}



\end{document}  