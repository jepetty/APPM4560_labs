\documentclass[11pt, oneside]{article}
\usepackage[margin=1in]{geometry}                % See geometry.pdf to learn the layout options. There are lots.
\geometry{letterpaper}                   % ... or a4paper or a5paper or ... 
%\geometry{landscape}                % Activate for for rotated page geometry
%\usepackage[parfill]{parskip}    % Activate to begin paragraphs with an empty line rather than an indent
\usepackage{graphicx}
\usepackage{amssymb}
\usepackage{amsmath}
%\usepackage{array}
%\usepackage{indentfirst}
%\usepackage{mathptmx}
\usepackage{enumitem}

\setcounter{secnumdepth}{3}

\title{APPM 4560 Laboratory 3 Report}
\author{Rhys Olsen\\
\texttt{rhys.olsen@colorado.edu}
 \and Jessica Petty\\
 \texttt{jessica.petty@colorado.edu}
 }
\date{November 16, 2016}

\begin{document}
\maketitle
\section{Simulating a Single Server Queue}
\subsection{Stationary Distribution of M/M/1-queue}
For an M/M/1 queue to have a stationary distribution, it is a necessary and sufficient condition that its arrival rate be less than its service. In the case of the queue provided, this means that $\lambda < \mu$.

\subsection{Distribution of a Stationary Queue}
When $X$ is stationary, this is equivalent to saying that it is in equilibrium. We know that an M/M/1-queue in equilibrium will have a geometric distribution of items inside. Now, given that $X_T$ is the number of items in the queue at time $T$, we can then say that the distribution of $X_T$ is geometric with success parameter $\frac{\lambda}{\mu}$.

\subsection{Busy Server}
When $X$ is stationary, the fraction of time that the server is \textit{not} busy is equivalent to the value $\pi(0)$. Conceptually, this is because the stationary distribution $\pi(x)$ represents the fraction of time that the queue has $x$ items in it. Clearly, if $x > 0$, this means the server is busy, as there are items in the queue, and when $x=0$ the server cannot be busy because there are no items in the queue, either waiting or being served.

For an M/M/1-queue, $\pi(0)=1-\frac{\lambda}{\mu}$. (PROOF can be found in notes if we decide it is necessary). Therefore, the fraction of time that the server \textit{is} busy is $1-\pi(0)=1-(1-\frac{\lambda}{\mu})=\frac{\lambda}{\mu}$

\subsection{Simulation of M/M/1-queue}
Hi Rhys, this is my best attempt at the algorithm. I figured that something written down was better than nothing, and you can swoop in and make it better.

\begin{enumerate}[leftmargin=30pt, labelindent=65pt, itemindent=30pt]
\item[\textsc{step 1:}] Given inputs $T$, $\lambda$, and $\mu$, set $t:=0$, $i_1:=0$, $D=[]$, $A=[]$, $AD=[]$
\item[\textsc{step 2:}] Simulate $r_1 \sim \text{Exponential} (\lambda)$ and set $t:=t+r_1$
\item[\textsc{step 3:}] If $t > T$, go to Step 4, else set $A[i]:=t$, $AD[i] = (t,1)$, $i_1:=i_1+1$ and go back to Step 2
%\begin{enumerate}[leftmargin=25pt, labelindent=65pt, itemindent=25pt]
%\item[\textsc{step 1.1:}] maybe
%\item[\textsc{step 1.2:}] yass
%\end{enumerate}
\item[\textsc{step 4:}] Set $t:=0$, $i_2:=0$
\item[\textsc{step 5:}] Simulate $r_2 \sim \text{Exponential} (\mu)$ and set $t:=t+r_2$
\item[\textsc{step 6:}] If $t>T$ go to Step 7, else set $D[i]:=r_2$, $i_2:=i_2+1$ and go back to Step 5
\item[\textsc{step 7:}] Set $i_{arr}:=0$, $i_{dep}:=0$, and $d:=A[i_{arr}] + D[i_{dep}]$
\item[\textsc{step 8:}] Set $i_{arr}:=i_{arr}+1$, $i_{dep}:=i_{dep}+1$
\begin{enumerate}[leftmargin=25pt, labelindent=65pt, itemindent=25pt]
\item[\textsc{step 8.1:}] If $d < T$, set $AD[i_1]:= (d,-1)$ and $i_1=i_1+1$, else return AD
\item[\textsc{step 8.2:}] If $d + D[i_{dep}] > A[i_{arr}]$, set $d:=d+D[i_{dep}]$ and go to Step 8
\item[\textsc{step 8.3}] Else if $d + D[i_{dep}] < A[i_{arr}]$, set $d:=A[i_{arr}] + D[i_{dep}]$ and to to Step 8
\end{enumerate}
\end{enumerate}
\subsection{}
\subsection{}
\subsection{}
\subsection{}
\subsection{}
\subsection{}



\end{document}  