\documentclass[11pt, oneside]{article}   	% use "amsart" instead of "article" for AMSLaTeX format
\usepackage[margin=1in]{geometry}               % See geometry.pdf to learn the layout options. There are lots.
\geometry{letterpaper}                   	% ... or a4paper or a5paper or ... 
%\geometry{landscape}                		% Activate for rotated page geometry
%\usepackage[parfill]{parskip}    		% Activate to begin paragraphs with an empty line rather than an indent
\usepackage{graphicx}				% Use pdf, png, jpg, or eps§ with pdflatex; use eps in DVI mode
						% TeX will automatically convert eps --> pdf in pdflatex		
\usepackage{amssymb}
\usepackage{amsmath}
\usepackage{array}
\usepackage{indentfirst}
\usepackage{enumitem}
\usepackage{mathptmx}
\usepackage{float}

%SetFonts

%SetFonts

\setcounter{secnumdepth}{3} % default value for 'report' class is "2"

\title{APPM 4560 Laboratory 2 Report}
\author{Rhys Olsen\\
\texttt{rhys.olsen@colorado.edu}
 \and Jessica Petty\\
 \texttt{jessica.petty@colorado.edu}
 }
\date{November 16, 2016}

\begin{document}
\maketitle
\part{Simulating a homogenous Poisson process (HPP)}
\section{}
The random variables $T(1), \dots, T(N)$ represent the times of the arrivals of the HPP on the domain $[0, t]$. Note that while $[0, t]$ is pre-determined by the way the problem is posed, $N$ is a random variable based on how many arrivals the simulation generates on $[0, t]$ and is \emph{not} pre-determined.
\section{}
As $N$ represents the number of arrivals of a Poisson process in a time window of size $t - 0 = t$, $N \sim \text{Poisson}(\lambda t)$.
\section{}
As $T(N)$ is defined as the time of the last arrival of the HPP in time interval $[0, t]$, $T(N + 1)$ represents the time of the first arrival of the HPP \emph{after} time $t$.
\section{}
The random variable $T(N + 1) - t$ is the amount of time taken after $t$ until the next arrival of the HPP. Each individual arrival of the HPP, including $T(N + 1)$, is exponentially distributed with rate paramater $\lambda$ by assumption. Since $N(T)$ is the \emph{last} arrival of the HPP on $[0, t]$, we have that $T(N + 1) > t$. By the memoryless property, $P(N(T + 1) > t + s | N(T + 1) > t) = P(N(T + 1) > s)$, which amounts to saying that $T(N + 1) - t \sim \text{Exp}(\lambda)$
\section{}
The random variables $T(N + 1) - t$ and $T(N + 1) - T(N)$ do \emph{not} have the same distribution.

Observe that $T(N)$ is the maximum among $N$ independent identically distributed uniform random variables on $[0,t]$ $U_1, \dots, U_N \sim_{\text{i.i.d.}} \text{Uniform}(0, t)$, so:
$$P(T(N) < t) = P(U_1 < t, \dots, U_N < t)$$
which by independence is:
$$P(U_1 < t) \times \dots \times P(U_N < t)$$
and as $$P(U_1 < t), \dots,  P(U_N < t)$$ are all identically distributed, for $U \sim \text{Uniform}(0, t)$, this is simply:
$$P(U < t))^n = \left( \int_{0}^{t} p_U(x) \mathrm{d}x \right)^n = \left(\frac{t}{t}\right)^n = 1^n = 1$$

Thus we always have that $T(N) < t \Rightarrow T(N + 1) - T(N) > T(N + 1) - t$.
\section{}

\part{Simulating a Non-homogenous Poisson process (NHPP)}
\section{}

\end{document}
